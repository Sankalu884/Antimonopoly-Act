\documentclass[11pt]{jlreq}

\usepackage[sect]{kian}
\usepackage{otf}
\usepackage{fancybox}
\usepackage{ascmac}
\usepackage{pxchfon}  
\setminchofont{A-OTF-RyuminPro-Light.otf}
\setgothicfont{A-OTF-FutoGoB101Pr6N-Bold.otf}



\title{\vspace{-30mm}{\textgt{\Large{共同の取引拒絶}}}}
\date{\vspace{-15mm}}


\begin{document}

\maketitle

\sectionB{}
	本件20社は、A社の運営会議において、P、Q、RにA社の共通乗車券を利用させないようにする(以下、「本件行為」ということがある。)旨決定し、実行している。このような行為は、競争者が共同して行う間接供給拒絶(2条9項1号ロ)に該当するのではないか。
	
	間接供給拒絶の要件は、「事業者が」、「正当な理由がないのに」、「競争者と共同して」、「他の事業者に」、「ある事業者に対する供給を拒絶させ」ることである。

\sectionC{「事業者が」、「競争者と」}
	本件20社はタクシー事業を営む事業者であり、かつ、甲市のタクシー需要者をめぐって競争関係にあると認められる。
\sectionC{「共同して」}
	「共同して」とは、意思の連絡を意味する。意思の連絡とは、複数事業者間で取引拒絶行為を行うことについて、相互に認識または認容して、これに同調する意思があることをいう。
	
	本問において、平成29年3月に開催されたA社の運営会議において、定額運賃のタクシー業者には、A社の共通乗車券を使用させないようにすることで意見が一致し、方針決定を行っているから、本件行為を行うことについて明示的に合意しているといえる。
	
	\sectionC{「他の事業者に、ある事業者に対する供給を拒絶させ」ること}
		A社は甲市におけるタクシーの「共通乗車券」を営む「事業者」であり、また、A社の共通乗車券を利用させないようにしたP、Q、Rは、いずれも個人タクシー「事業者」である。
		
		本件20社がA社に行わせているのは、共通乗車券事業に係る契約の解約及び新規の申込の停止であるから、共通乗車券事業に係る役務の「供給」の拒絶である。
		本件20社は方針決定を行うことによって、A社とP、Q、Rとの取引を拒絶させているから、本要件を充足する。
	
	\sectionC{「正当な理由がないのに」	}
		「正当な理由がない」とは、公正競争阻害性を意味する。共同の取引拒絶においては、代替的取引先の発見が困難になり、被拒絶者の事業活動が困難になる蓋然性が高いため、市場閉鎖を通じた自由競争減殺が発生することがその内容である。
		
		\sectionD{市場画定}
			自由競争減殺の生じる市場の画定を行う。本件行為は、共通乗車券を利用させないようにするものであるから、これを起点に需要者からみた代替性及び、補充的に供給者からみた代替性を考慮する。
			
			まず、地理的範囲については、甲市がタクシー事業について独立した交通圏を形成していることから、甲市内と画定できる。
			
			共通乗車券は、降車時に自ら支払うことなくタクシーを利用できるという特性があり、官公庁・企業等の法人がその特性を重視して経理上の便宜等の観点から利用するものであって、需要者からみて代替性はない。
			また、共通乗車券が利用できるタクシーは乗客の獲得上有利であり、また、甲市におけるタクシー事業者の大半は小規模事業者であって、A社以外に同様の共通乗車券を営むものは存在しないから、供給の代替性も認められない。
			
			もっとも、タクシー事業には、共通乗車券を使用する場合と、通常通り料金をその場で支払う場合の2つの供給経路がある。そこで、SNIPP基準を用いて判断する。
			甲市において共通乗車券事業を行っているのはA社のみであり、A社が小幅であるが、実質的かつ一時的でない価格引き上げを行った場合、官公庁や企業等は、共通乗車券の上記特性を重視して利用しているため、通常通り料金をその場で支払う業態のタクシーの利用に移行するとは考えがたく、A社は利潤を拡大することができる。
			
			以上により、市場は、「甲市における共通乗車券を使う乗客向けのタクシー運送役務の市場」と画定できる。
			
		\sectionD{市場閉鎖効果}
			本問において、共通乗車券事業に参加している事業者は、本件20社及び5社であり、それらが保有するタクシーの台数は甲市で稼働するタクシーの9割以上を占めており、市場における地位は相当に大きい。また、共通乗車券事業を行っているのはA社のみであり、P、Q、Rにとって代替的取引先は存在しない。
			
			さらに、現在本件行為はP、Q、Rに対してのみ行なわれているが、今後、低額運賃を適用して本件市場に参入しようとする事業者に対しても行なわれうるものであり、これにより、人為的に参入を阻止することができると考えられるので、市場閉鎖を通じた自由競争減殺が認められる。
			
		\sectionD{正当化事由}
			もっとも		正当化事由が認められれば、公正競争阻害性が認められないこともありうる。
			本問でも、低額運賃による客の奪い合いが、歩合制で働く乗務員の収入減および過重労働を強いて、交通事故の増加につながる危険があるとの懸念があるとしているから、一般消費者の利益確保に向けた行為として、正当化事由が認められないか。
			
			正当化事由として認められるためには、\UTF{2460}目的が正当であり、\UTF{2461}目的に照らして合理的に必要な範囲の、\UTF{2462}相当性の認められる行為であることを要する。
			
			本問において、歩合制で働く乗務員の収入減および過重労働を強いて、交通事故の増加につながる危険を防止するという目的は正当であるといえる。しかし、過重労働・交通事故を防止するために本件行為を行うことが必要であるとはいえず、合理性が認められない。現に、P、Q、Rはすでに低額運賃でのタクシー事業を2年以上継続しているにもかかわらず、そのような危険が現実化しているとの事情は認められない。
			
			したがって、正当化事由は認められない。
			
	\sectionB{以上により、本件行為は「正当な理由なく」行なわれている競争者が共同して行う間接供給拒絶であり、2条9項1号ロに定める不公正な取引方法に該当する行為である。}

\begin{flushright}
	以上
\end{flushright}
	
\end{document}










